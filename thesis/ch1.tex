%   MSc Business Analytics Dissertation
%
%   Title:     The Relationship Between Corporate Governance and Company Performance
%   Author(s): Conor Reid
%
%   Chapter 1: Introduction and basic definitions
%
%
\chapter{Introduction}\label{C.intro}
{This chapter forms the introduction to the current study, laying out its motivations and aims. A brief background of the domain is given, followed by the business motivation for this study. This is followed by a summary of the academic contribution, the research goals and scope and ends with an outline of the document structure as a whole.  }
\section{Background}
{This study is primarily concerned with the relationship between corporate governance and company performance, particularly with how the former can be optimised to positively influence the latter. Corporate governance is a widely discussed, debated and researched topic that is as relevant today as it has ever been. A companies governance structures dictates it's policies and actions, ensuring that all stakeholders \footnote{A stakeholder is someone who has a stake, or a personal interest, in the company. Employees, the local community and the media are all stakeholders.} have input into how the company is run facilitating a shared vision of where the business it is going. Governance policy also acts to mitigate financial and ethical pitfalls by setting clear standards. Thus it is fair to say that corporate governance has a wide ranging influence within and also outside the company.\\\\
We frequently see instances of corporate governance failure, which can lead to disastrous consequences both financially and reputationally. Reputation is often of high importance in both the public and private sectors, which motivates more ethical and fair behaviour in order to protect it. Instances where companies fail in this regard often make eye-catching  headlines, for example \cite{kirkpatrick2009corporate}, \cite{telegraphVolkswagon} and \cite{cnnUnitedAirlines}. In a hyperconnected world where news spreads quickly, the importance of a functioning governance structure is more important than ever. At the time of writing California have imposed a quota on female presence in the board room, outlined by \cite{ftWomenBd}. This highlights the perceived importance of governance practices, and the willingness of legislators to enforce polices they believe to be beneficial to the running of any business in todays society.\\\\
The interests of different parties can conflict in any business, such as between the shareholders \footnote{A shareholder is often an investor who has equity in the company. They often have no personal interest in the company, solely financial.} or directors. There is much debate on how best to align these interests, with suggested initiatives like structuring executive compensation to be at least partly dependant on firm performance. Shareholder interests can also conflict with the interests of the wider public as a whole, or the stakeholders. This is especially true for companies that heavily rely on natural resources as a driver for growth. In this case, sustainability not just of the company but of finite natural resources that the public depend on must be closely governed and managed. This is often the responsibility not just of those within the company, but those outside it too.  \\\\
It is reasonable to argue that corporate governance influences all aspects of the company, not least its economic success. \cite{moldovan2015learning} studied this relationship, collecting data on corporate governance and using it to predict corporate success as measured in various ways. They were able to learn models that did this successfully, resulting in a number of rules that inform corporate governance best practice. The current study uses this as a starting point, and looks to address some limitations within, as outlined in section \ref{RGAS}. Modern work around causation is also studied, with a view to applying it in this domain. This would act to strengthen previously derived relationships, and presents an opportunity to study the deeper causal influencers of corporate economic outcomes.}
\section{Business Motivation}
{\cite{moldovan2015learning} state the conclusions they reached in their research in the form of simple {\it if-this-then-that} styles rules, and point to the business significance of each. For example, they found that for US based companies the number of women on the board of directors was positively related to company performance. They also found that in Western Europe, companies should employ larger audit teams that has a mitigating effect on the risk of bankruptcy. In Eastern Europe, their main finding is that the presence of an independent chairman best influences economic success.\\\\ 
The business benefit of the above, and other similar findings, is obvious. By deriving a number of relationships between economic success and corporate governance, the authors first prove that a relationship does in fact exist in the first place. That is, elements of high corporate governance performance are strongly associated with company outcomes. Secondly they are able to put forward recommendations for governance best practice and show what elements are most influential, with geographic context. A key element of management is identifying levers with which to effect outcomes in a positive way, providing the core motivation for this type of research.\\\\
We look to first verify the above findings, before expanding the scope to include elements of governance not considered in the original study. This would in effect expand the array of tools available to corporations for effecting economic change. We also aim to strengthen these findings by seeking to identify causal mechanisms within this domain,  and thus provide a more informed platform for decision makers. }
\section{Academic Contribution}
{A key element of this study is the exploration of causality research and the application of these techniques in this domain. There is continual active research in this area, with interested parties offering new techniques and methodologies for making the step from correlation to causation in a variety of domains. To our knowledge, casual research has not be applied in the area of corporate governance and its effect on outcomes, and thus would represent a novel endeavour that stands to contribute to the field in a meaningful way. \\\\ 
For example, the rules proposed by \cite{moldovan2015learning} are backed by strong correlations drawn from highly accurate statistical models. However they make no steps towards estimating a cause and effect element to those relationships, or any other type of deeper analysis. We propose that a significant academic contribution would be to explore how causality is reached and applying that methodology here, to see if more can be said of the aforementioned rules. \\\\
As mentioned above, this study plans to expand the work of \cite{moldovan2015learning} to include other types of company actions and activity. This would help gain a more holistic view of how economic success can be promoted across all company functions. }
\section{Research Goals and Scope}{\label{RGAS}}
{There are a number of key goals that this study aims to achieve. They are presented below, along with a discussion of how success will be measured at each stage.
\begin{enumerate}
\item{\bf {Reproduce the findings of \cite{moldovan2015learning}}.}\\
{As mentioned, \cite{moldovan2015learning} made findings that point to interesting relationships between corporate governance and company performance. It would be useful to use similar data to reproduce some of these findings using the same techniques as the authors. Measuring success here is relatively straightforward, facilitied by simple model comparison across a number of performance related metrics. }
\item{\bf {Remodel the problem}}\\
{\cite{moldovan2015learning} thresholded on the various measures of company success, that are continuous in nature. This frames the problem as a classification exercise whereas a more natural framing is as a regression exercise. Thus, the study aims to use the same data as before but neglect the thresholding step and perform a regression on the continuous measures of success. By doing so, this study aims to build models that better reflect the true nature of the data. This stage also includes integrating additional independent and dependent variables, to expand the scope and discover new relationships between corporate governance and economic outcomes.    }
\item{\bf {Apply modern work on causality.}}\\
{A number of conclusions on the influence of corporate governance on company performance were reached, using established statistical analysis to subsequently discover various correlations. In order to strengthen these findings and gain deeper insight into the underlying mechanisms of the domain, modern work in causality will be applied. This will involve significant research into the ways in which this can be achieved, including data requirements and any pre-processing steps required. The aim here is to gain a much deeper understanding of the causal structures the effect corporate economic success, to drive best practice and contribute to knowledge base in this area. }
\end{enumerate}
{It is equally important to discuss what is out of the scope of this study. This study considers only public companies, since the data that has been obtained related to companies listed on various stock exchanges. The study would be complicated by the inclusion of private companies, due to regulation differences and enforced governance structures.\\\\
Further, regulatory differences from country to country are not considered. The financial reporting standards are not completely consistent across the regions included in this study, however for the purposes of this study these differences are not accounted for. Some countries may also introduce certain taxation or laws that influence the decisions made by companies in those regions, like a carbon emissions tax that may make companies take their environmental footprint more seriously. Again, no effort to account for these differences is taken.} 
}
\section{Document Outline}
{This document is laid out as follows. Chapter \ref{C.LitReview} contains a literature review of this topic including how corporate success can be measured, other predictive corporate features that may be included, a review of other similar studies and concludes with a summary of research in the area of causation. Chapter \ref{C.Methodology} contains details of this studies methodology, including a summary of the data used and its pre-processing, algorithms used and methodology around applying causal techniques. Chapter \ref{C.Results} contains the results of this study. Included in chapter \ref{C.Discussion} is a discussion of these results, with some concluding remarks and opportunities for future research outlined in chapter \ref{C.Conclusions.Future.research}.}
