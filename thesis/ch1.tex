%   MSc Business Analytics Dissertation
%
%   Title:     Aaa Bbbbbbb Cccccccccc
%   Author(s): Xxxxxx Xxxxxxxxx and Yyy Yyyyyyyyy
%
%   Chapter 1: Introduction and basic definitions
%
%   Change Control:
%   When     Who   Ver  What
%   -------  ----  ---  --------------------------------------------------------------
%   11Feb11  AB    0.1  Begun
%
\chapter{Introduction}\label{C.intro}
{This chapter forms the introduction to the current study, primarily laying out its motivations and aims. We begin with a background of the domain, followed by the business motivation for this study. We follow this with a summary of the academic contribution, the research goals and scope and end with an outline of the document structure.  }
\section{Background}
{This study is primarily concerned with the relationship between corporate governance and company performance, particularly with how the former can be optimised to positively influence the latter. Corporate governance is a widely discussed, debated and researched topic that is as relevant today as it has ever been. We frequently see instances of corporate governance failure, which can lead to disastrous consequences. Reputation is often of high importance in both the public and private sectors, which can promote the importance of ethical and fair behaviour. Instances where companies fail in this regard often make attractive headline news stories. In a hyperconnected world, the importance of a functioning governance structure is more important than ever. \\\\
In any business, the interests of shareholders \footnote{Or {\it stakeholders}, those with a legitimate interest in the company but not necessarily those with any formal ownership therein.} can conflict with those of directors. There is much debate on how best then to align interests, for example by structuring executive compensation to be at least partly dependant on firm performance. Shareholder interests can also conflict with the interests of the wider public as a whole, which is especially true for companies that heavy rely on natural resources as a driver for business. In this case, sustainability not just of the company but of finite natural resources must be closely governed and managed by those both inside and outside the company.     \\\\
It is reasonable to argue that corporate governance influences all aspects of the company, not least its economic success. \cite{moldovan2015learning} studied this relationship, collecting data on corporate governance and using it to predict corporate success as measured in various ways. They were able to learn models that did this successfully, resulting in a number of rules dictating governance of high performing companies. The current study uses this work as a starting point, and looks to address some limitations within by using alternative measures and techniques. These are outlined in section \ref{RGAS}. Modern work inferring causation is also studied, and sought to be applied in this domain in order to strengthen correlative relationships.   }
\section{Business Motivation}
{\cite{moldovan2015learning} state the conclusions they reached in this space. For example, they found that for US based companies the number of woman on the board of directors was positively connected to company performance. They also found that in Western Europe, companies should employ larger audit teams that in turn lowers the risk of bankruptcy. In Eastern Europe, their main finding is that an independent chairman best influences economic success.\\\\ 
The business benefit of the above is obvious. By deriving a number of relationships between economic success and corporate governance, the authors first prove that a relationship does in fact exist in the first place. Secondly they are able to put forward recommendations for governance best practice and show what elements are most influential, with geographic context. A key element of management is identifying levers with which to effect outcomes in a positive way, which leads into the motivation for this research.\\\\
We look to first verify some of the above findings, but also look to find new influencers of economic success by expanding the research to include other predictors. This would in effect expand the array of tools available to corporations for effective governance and business practise. We also aim to strengthen these findings by seeking causal influencers, which would add a hierarchical element to the range of levers for change and direct efforts to spaces that are most likely to yield success. }
\section{Academic Contribution}
{A key element of this study is the exploration of causal research and the application of these techniques in this domain. There is continual active research in this area, with interested parties offering new techniques and thought processes for making steps towards proving causation in a variety of domains. To our knowledge, this type of research has not be applied in the area of corporate governance and its effect on outcomes, and thus would represent a novel endeavour that stands to contribute to the field in a meaningful way. \\\\ 
For example, the rules proposed by \cite{moldovan2015learning} are backed by strong correlations drawn from highly accurate statistical model. They make no steps towards estimating a cause and effect element to those relationships, or any other type of deeper analysis. We propose that a significant academic contribution would be had by exploring how causality is reached and applying it here, to see if more can be said of the aforementioned rules. This also applies to any other rules that can be derived by expanding the research to include other potentially predictive corporate behaviours and actions. }
\section{Research Goals and Scope}{\label{RGAS}}
{There are a number of key goals that this study aims to achieve. They are presented below, along with a discussion of how success will be measured at each stage.
\begin{enumerate}
\item{\bf {Reproduce some of the findings of \cite{moldovan2015learning}}.}\\
{As mentioned, \cite{moldovan2015learning} made some interesting findings that point to interesting relationships between corporate governance and company performance. It would be useful to use similar data to reproduce these findings using the same techniques as the authors.}
\item{\bf {Improve on these findings}}\\
Next, the aim is to improve on these results using three methods. The first involves considering other predictors of corporate success, such as a company's social responsibility performance or their environmental impact. The second involves using alternative measures of corporate success, that may better reflect how successful a company is. The measures used by \cite{moldovan2015learning} leave room for improvement. The third method involves using alternative statistical techniques with a view to improving model performance using the standard measures of model accuracy. The way in which data is preprocessed may also be altered as part of this step. For example, the authors discretise corporate success and perform classification. It may be advantageous to perform regression analysis here to gain greater granularity.
\item{\bf {Apply modern work on causality.}}\\
{A number of conclusions on the influence of corporate governance on company performance have been reached, using established statistical analysis and subsequently discovered correlation. In order to strengthen these findings and potentially find new relationships, modern work in causality will be applied on similar data in this domain. This will involve significant research into the ways in which this can be achieved, including data requirements and required pre-processing. The aim here is to gain a much deeper understanding of the casual influencers of corporate economic success, to drive best practice and contribute to knowledge base in this area. }
\end{enumerate}
\subsection{Out of Scope}
{A distinction is not made in this study between public and private companies, although regulations dictating how public companies must govern are often more stringent and strictly enforced than private companies. Privately held forms often how more freedom and flexibility here. This can be especially true when dealing with audits and so on. \\\\
Further, regulatory differences from country to country are not considered. For example, some nations may introduce certain taxation and laws that influence the decisions made by local companies, like a carbon emissions tax that may make companies take their environmental footprint more seriously.} 
}
\section{Document Outline}
{This report is laid out as follows. Chapter \ref{C.LitReview} contains a brief literature review of this topic including how corporate success can be measured, other predictive corporate features that may be included, a review of other similar studies and concludes with a summary of research in the area of causation. Chapter \ref{C.Methodology} contains details of this studies methodology, including a summary of the data used and its pre-processing, algorithms used and methodology around the casual work. Chapter \ref{C.Results} contains the results of this study. Included in chapter \ref{C.Discussion} is a discussion of these results, with some concluding remarks and opportunities for future research outlined in chapter \ref{C.Conclusions.Future.research}.      }
