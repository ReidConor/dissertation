%   MSc Business Analytics Dissertation
%
%   Title:     Aaa Bbbbbbb Cccccccccc
%   Author(s): Xxxxxx Xxxxxxxxx and Yyy Yyyyyyyyy
%
%   Chapter 7: Conclusions and Future Research
%
%   Change Control:
%   When     Who   Ver  What
%   -------  ----  ---  --------------------------------------------------------------
%   11Feb11  AB    0.1  Begun 
%

\chapter{Conclusions and Future Research}\label{C.Conclusions.Future.research}

\section{Introduction}\label{S.Concl.intro}
{This chapter contains some concluding remarks about this project and in doing so draws together its aims, methodology and results. As with the other chapters in this document, the replication and causal estimation stages of this study are discussed separately although reference is made to both at times. This chapter finishes with a discussion of future work in this space that would likely yield interesting insight. }
\section{Replication}
{Overall, the results of the classification stage of this project were very positive. The preliminary aim here was to implement a subset of algorithms used by \cite{moldovan2015learning} on identical data, with a view to replicating and verifying those findings. As shown by the various tabulated data in section \ref{S.classification4}, this was certainly achieved. It is evidently possible and viable to model the relationship between corporate governance (in conjunction with other characterising corporate features) and economic outcomes, with a high degree of accuracy. In relation to the Altman Z score, some results were suboptimal. Specifically, the models ability to identify companies in the "grey" risk band was poor compared with other bands. As mentioned before, merging risk bands to create a binary "safe" and "not safe" dependant variable (as was done for the causal estimation stage) would likely address this issue. One could also engineer that dependant variable in a different way, for example a "high risk" and "not high risk" class variable. A decision would need to be made as to the requirements of the user. \\\\
This study was able to successfully remodel the original research question as a more natural regression analysis, presenting data in section \ref{S.regression4}. The accuracy achieved by these models is mixed. In the classification stage performance metics across the board are higher when classifying the Tobin's Q score than the Altman Z score, and this is a trend that continues here for the regression models. For Tobin's Q, $r^2$ values for the S\&P and STOXX Eastern Europe 300 datasets are relatively high at around 0.72 with low respective RMSE values. It could be reasonably argued that these are useful models. Perhaps less useable is the regression model built using the STOXX Europe 600. For both dependant variables, $r^2$ values are reasonable but the RMSE is very large, meaning the values of the predicted and actual outcomes are very different. It could be that this model is severely overfitting leading to a high degree of explained variation coupled with a high degree of prediction error. Again, the requirements of the user must be accounted for here. In some cases a model that explains the data very closely is most useful, in others it is more important to be able to predict the Q or Z score for some as-yet unseen data. \\\\
Finally, this study shows it is possible to model the relationship between corporate governance and the Benish M Score, which measures the probability of a firm intensionally manipulating their financial reportings. Using the eight variable construction of this measure proved optimal, yielding a comparable $r^2$ as the five variable construction but with an RMSE of nearly half. The $r^2$ is relatively low at approximately 0.25, meaning the model does not explain a large proportion of the variation in the dataset. It is likely that a more purpose built dataset would be required to address this issue, that might include other independent variables covering a wider range of company activity. Having said this, attaining model accuracy of note at all in this case can be seen as an interesting achievement.   }
\section{Causal Estimation}
{The causal estimation portion of this study has returned some interesting results. As mentioned previously, the reliability of the average treatment effect (ATE) results relies heavily on the quality of the matching process. For the true effect of the treatment to be identified, there needs to be a large quantity of characteristically equivalent units (companies in this case) that differ only on treated status. Matching plots with significant overlap on the x-axis show that matching quality is high, with poor overlap indicating that the control and treatment groups are characteristically different. generally very good matching can be seen across result sets presented in section \ref{S.causal4}, adding validity to the associated ATE. Control variables such as \texttt{Asset}, \texttt{ROC} and \texttt{Net Debt/EBITDA} all consistent facilitate accurate data stratification on treated status. Conversely, variables such as \texttt{Fin Lev} and \texttt{Interest} are shown to have poor population overlap which likely degrades the overall matching quality. The variables chosen to control on for each dataset represent the most important variables as per a classification model that uses the treatment status as a binary dependant variable. It may be that picking the top $n$ variables from that model does not lead to optimal results. With more time, a more in-depth investigation could be carried out to answer this question. \\\\
It is clear that the present study has been able to identify which statements made by \cite{moldovan2015learning}, who based their {\it if-this-then-that} style rules on identified correlation, have causal merit and which don't. A total of six original statements were tested using the causal estimation framework presented in this study. Of those six, three findings agreed with the original statement and three did not. Agreement here means that the sign of the ATE matched the direction of the correlation found in the original study, regardless of magnitude. To those statements, the current study added two more which utilised auxiliary corporate governance features appended to the original dataset, and were based on various findings in the literature. Both ATE's in those cases agreed with the literature and act to strengthen those findings beyond simple speculation or correlation.   }
\section{Future Research}
{There is significant scope for extending this study beyond the bounds imposed here, in a number of directions. Firstly, stock indexes outside of those used here could be easily integrated and would facilitate a wider analysis of the characteristics of the relationship between corporate governance and company performance. This study considers the S\&P 500, which along with the NASDAW Composite and Dow Jones Average make up the three most followed indices in the United States. Including these indices may better characterise the variation in governance practices employed by American companies. Asian stock indices would also be a worthy addition. For example the largest stock index in China, the Shanghai Stock Exchange Composite Index, is included in the Bloomberg financial data repository and would significantly broaden the scope of this study.  \\\\
As well as other major indices that track some of the most successful companies in the world, this study would be strengthened by including companies with lower market capitalisation. The Russell 2000 index for example tracts approximately 2000 small-cap companies and acts as a benchmark for small-cap stocks in the United States. Including such companies in this study would enable more nuanced insight.   \\\\
Missing data was an issue in this study, requiring some analysis and eventually a work-around. This missing data may become available at a future date, whether within the Bloomberg ecosystem or some other data repository. Integrating such data would be beneficial, coupled with an analysis of mandatory reporting practices within each market to determine whether data is missing at random or not. \\\\
As discussed in chapter \ref{C.LitReview}, a temporal aspect to the data is often employed in causal studies. This allowed the tracking of performance over a period of time, in which one or many interventions may have occurred enabling analysis of performance before and after to establish an effect. Bloomberg makes such historical data readily available.\\\\
Finally, moving away from dataset specific enhancements, an area for future research is on the more technical side of this study. As mentioned in section \ref{methodCausal}, the variables to control on for the causal estimation stage of this study were selected as the variables that were shown to be important in classifying the treatment variable. A more robust method of variable selection would be to use the Back-Door Criterion, discussed at length by \cite{pearl2009causality}. This is a method for controlling for confounding bias, and may provide a more valid set of covariates to control for in the propensity score matching phase. }