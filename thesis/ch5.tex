%   MSc Business Analytics Dissertation
%
%   Title:     Aaa Bbbbbbb Cccccccccc
%   Author(s): Xxxxxx Xxxxxxxxx and Yyy Yyyyyyyyy
%
%   Chapter 6: Discussion
%
%   Change Control:
%   When     Who   Ver  What
%   -------  ----  ---  --------------------------------------------------------------
%   11Feb11  AB    0.1  Begun 
%

\chapter{Discussion}\label{C.Discussion}
\section{Introduction}\label{S.Discussion.intro}
In this chapter a discussion of the results presented in chapter \ref{C.Results} is included. This involves discussion around the key findings and insights gained, drawing on the findings of \cite{moldovan2015learning} and others for comparison where applicable. Thought is given to cases where the results contradict previously held assumptions, drawing on surrounding literature and analysing the experimental method used. \\\\This chapter is split into two parts. The first looks at results for the replication stage of this study which encapsulates the classification and regression exercises. The second discusses the results of the casual estimation stage in detail. 
\section{Replication}\label{S.Discussion.replication}
%\subsection{Across the Board}
Overall, model performance for the current study is very close to that of the work of \cite{moldovan2015learning}. With few exceptions, the accuracy achieved and precision for each class as well as the ROC area match very closely. This is to be expected, since we are using a near identical dataset and methodology. There are some notable exceptions to this, where the current study makes some improvements. For the S\&P 500 dataset for example, model accuracy is marginally higher across all algorithm implementations and dependent variables. Accuracy for the J48 (C5.0) algorithm using Altman Z score for instance is approximately 5\% higher. This is driven by an increased ability to correctly identify ``safe'' levels of risk. Having said this, the ROC area is slightly smaller here which highlights the decreased ability to identify other levels of risk. In this case, the user of these results needs to decide which metric is most important given their situation. \\\\
%\subsection{Adaboost vs j48, tobin vs altman}
A significant difference in the ability to classify the Tobin's Q score and Altman Z score is clear. In both the original and current study, model performance across all metrics are generally higher for Tobin's Q than Altman Z. This could be due to the complexity of the problem. The Altman Z score contains three classes, with performance in identifying the middle class particularly poor. Across all markets and algorithms for the Z score, we see a large decrease in performance for class 1 (denoting ``grey'' risk) compared to class 0 (``distress'' risk) and class 2 (``safe'' risk). This is not an unreasonable finding, since the problem of identifying extremes is much easier than performance levels in units that are inherently in-between. Classifying Tobins Q in comparison is much simpler of a problem, due to its two class construction. It is likely that if the Altman Z score thresholds were altered (merging ``distress'' and ``grey'' for example) then model performance would improve at the cost of the nuanced insight inherent with a more complex model.\\\\
There is significant difference too between the levels of performance of each algorithm. Adaboost M1 in general achieves significantly higher levels of accuracy and ROC area than J48 (which for the purposes of the current study, is in fact C5.0). This is a finding that mirrors the original study, although no explanation is given there as to why there is such a difference. It may be that the learner used in Adaboost makes assumptions or employs efficiencies that are optimal for these datasets, although typically empirical research is used to determine the {\it best} algorithm for the particular problem with little thought as to why that might be. This is reflected in the work of \cite{moldovan2015learning} who do exactly this. \\\\
%\subsection{Removing outliers}
As mentioned in chapter \ref{C.Methodology}, \cite{moldovan2015learning} remove outliers before the modelling stage. They cite ``data errors'' as the cause of outliers, although neglect a more in-depth analysis of the impact of such an omission. Again mentioned before, modelling in the current study is carried out with and without outliers present as an empirical investigation of their impact. The results of this investigation are inconclusive overall, although performance benefits are seen in some cases when outliers are included. For example with the S\&P dataset, accuracy and ROC area are higher in the presence of outliers for both dependent variables and both algorithms. Thus it can be argued that outliers in this dataset are important for accurately characterising corporate governance styles and company performance. \\\\The same cannot be said for the other two markets, the STOXX Europe 600 and STOXX Eastern Europe 300. With some minor exceptions, the inclusion of outliers tends to decrease model performance, although in some cases performance still betters the original study. In these cases then, it could be argued that the authors omissions were valid and acted to improve results.  \\\\
%\subsection{Adding extra variables}
Model performance is also impacted by the addition of auxillary corporate governance related features, as displayed for the S\&P index. These additions increased accuracy by as much as 6\% over the original findings (Adaboost on the Altman Z score) with marginal differences across the other metrics. This shows that those variables are valid inclusions, and justifies their use later in this study. Regardless of any improvement or not, the fact that performance didn't decrease validates their inclusion. This is further supported by the example decision tree shown in figure \ref{sampleDT}. The social disclosure score, a newly integrated feature, is shown to be an important variable in the classification of the Altman Z score, and is listed as having an attribute usage of 5\%. \\\\
%\subsection{Regression}
The regression analysis carried out in this study is an extension of the original work of \cite{moldovan2015learning} as so these results standalone. Overall, it is clear that the ability to accurately model variation varies across these datasets. As does the choice of lasso and ridge regression, each of which is shown at times to yield better accuracy in different situations. 
\\\\
For the S\&P 500, the $r^2$ values vary too with dependent variable. Regressing on Tobin's Q yields an optimal value of approximately 0.74 with a RMSE of approximately 0.76. Note here that RMSE is in the same units as the dependent variable. These values are found at an $alpha$ value close to lasso regression, where variable coefficients are reduced to zero where possible. Interestingly, there are points along the $alpha$ spectrum that yield significantly reduced levels of performance, indicating that the choice between lasso and ridge (and indeed, elastic-net) is non-trivial. Having said this, in general performance is fairly uniform across the spectrum. This indicates that simplifying the model by reducing coefficients to zero, as lasso does, does not have to mean a reduction in model quality. The optimal $r^2$ for Altman Z score as the dependent variable is significantly worse, at approximately 0.51 with an RMSE of approximately 9.61. This adds validity to the results seen above at the classification stage, where the algorithms performed poorly at identifying companies with middling levels of bankruptcy risk. It is likely that the inherent ambiguity of the Altman Z score may be leading to these results. \\\\
Similar results are found for the STOXX Europe 600 index. The model here achieves an optimal $r^2$ of a very high 0.92 for the Tobin's Q score, albeit with a significantly higher RMSE than the S\&P 500 of 4.52. Such a high level of RMSE likely renders this model unusable and so effort would need to go into altering this model appropriately. Results for the Altman Z score show a similar drop as seen before, with an $r^2$ of 0.49 and a RMSE of 12.4.\\\\
For the STOXX Eastern Europe 300 index, similar results to the S\&P 500 are seen. For Tobin's Q score, an $r^2$ of 0.73 with an RMSE of 0.48 is achieved by using a penalty factor close to that employed by ridge regression. This might imply that there are numerous variables that are useful for classifying the Q score, and reducing some of them to zero as lasso does is suboptimal. For the Altman Z score, an $r^2$ of 0.52 and an RMSE of 8.25 is shown, following a similar trend of decreased accuracy from Tobins Q score.\\\\
Looking back now at the S\&P 500 regression, interesting results can be seen for the Benish M score results. Firstly, it is clear that neither ridge nor lasso regression is optimal here. In practice one might opt for lasso, since the models there are more likely to be more simple (that is, include less covariates and thus less complexity). The $r^2$, while low, is still significant at approximately 0.25 for the eight variable construction and 0.3 for the five variable construction. 
\section{Causal Estimation}\label{S.Discussion.causal}
The results pertaining to the causal estimation stage of this project are broken down by market, and then by treatment and effect pair. Each is based on a different motivating statement found either in the original work of \cite{moldovan2015learning} or some other piece of work in the literature on this topic. This section of the discussion will proceed in a similar fashion, referring to each in isolation with reference to the expected and actual outcome as well as the quality of the matching rate. 
%based on statement 2, reiterate the statement? 
%agrees with that statement, fairly strongly. tight enough CI
%what does the \% change mean? 
%so the outcome here is a binary class 
%so the \% is the increased \% chance that the treated unit will be in the other class?
%ie 15\% more likely to be in the other class, 
%here "other" is dictated by the sign of the estimate
\subsection{S\&P 500}
\subsubsection{Independent Director \& Financial Leverage}
{The first result set for the S\&P market is based on the second motivating statement listed in section \ref{CausalEstimation-ResearchQuestions}, which states that for American companies, the presence of an independent lead director in addition to a financial leverage higher than 2.5 incurs a lower Altman Z score. These features are thus used in the construction of the treatment, using the discretised Altman Z score as the outcome. We expect the average treatment effect (ATE) to be minus in sign, which it does turn out to be. The effect of this treatment is an increase in the probability of treated units falling into the ``not safe'' category of the Z score by a factor of between 9\% and 15\%. The magnitude of this effect is relatively significant, and adds validity to the original statement. }
\subsubsection{CEO Compensation}
{The total CEO compensation as an independent variable was introduced in the current study, and so no prior results are available for comparison. Instead this result set is based on statement \ref{wildOne}, derived from the literature, which states that firms with weaker corporate governance tend to underperform and also tend to reward CEO with greater levels of compensation. We expect here then to see a negative ATE, which is indeed present. The magnitude of the effect is between 6\% and 11\%, again a fairly significant score. The quality of matching is high, which significant overlap on the x-axis indicating a good rate of comparable control and treatment units across control variables. }
\subsubsection{Board of Directors Average Age}
{Moving now on to the effect of the average age of the board of directors, it should first be noted that the motivating statement for this result set was originally tied to the STOXX Eastern Europe 300 stock index. However, it was prohibitively difficult to achieve a sufficiently good matching rate with that dataset in this context. As can be seen here though, the quality of matching is good with enables the testing of this statement regardless.\\\\ The expectation here is that the treatment, which is present if the average age is less than the mean, causes a positive ATE. However the opposite is true here, with an ATE of between -6\% and -10\%. This indicates that in fact an older and potentially more mature board tends to increase company performance as measured by Tobins Q. This is an interesting finding despite the fact that the original statement pertains to a different market.}
\subsubsection{Social Disclosure Score}
{The final result set for the S\&P 500 dataset relates to the social disclosure score, another new independent variable introduced in this study. The motivating statement for this result set is statement \ref{wildTwo}, which states that strong performance in terms of environmental and social considerations is positively correlated with firm valuations and by association, performance. As expected, the ATE is positive in nature which adds validity to this statement. Important to note here is the spread of the 95\% confidence interval, which estimates the true value to lie within a 10\% window. This is relatively high, especially when compared to other result sets, and reflects an uncertainty as to the true magnitude of the effect of this treatment. The direction of the effect can be said with higher certainly however, which is certainly a useful piece of insight. }
\subsection{STOXX Europe 600}
\subsubsection{Female Board Membership}
{Moving now to the STOXX Europe 600 dataset, the first result set to discuss revolves around the presence of women on the board of directors. This research question is tied to two motivating statements, originating from \cite{moldovan2015learning}, which both offer opposing effects of this treatment for different markets. Statement \ref{westTwo} states that this treatment should negatively effect performance in this market. Statement \ref{spOne} however states that for American firms, this same treatment should be positivity related to performance. Interestingly this result set agrees with the latter, despite it using data from the former. The ATE is positive, and estimates an effect of between 6\% and 14\%. The matching quality is high, with very significant overlap along the x-axis for each control variable.}
\subsubsection{Independent Director or Former CEO on the Board}
{Beginning a trend of findings that disagree with their respective motivating statements, the next results set relates to the presence of an independent director or former CEO on the board. Statement \ref{westOne} states that such a condition should be negatively related to company performance as measured by Tobins Q score. This is contradicted here, with a massive ATE of between 30\% and 40\% in the positive direction. Again the confidence interval is relatively large here when compared to others, however a true ATE anywhere within this interval should be consider substantial. This is backed by a strong level of matching between the control and treated groups. }
\subsubsection{Independent Director \& Financial Leverage}
{The final result set for the STOXX Europe 600 index revolves around the presence of an independent director and a large degree of financial leverage. The formulation of this result set is based on statement \ref{spTwo} which in fact relates to the American market. \cite{moldovan2015learning} state that this treatment tends to increase bankruptcy risk as measured by the Altman Z score. This is reflected in the ATE found here, which is negative indicating a risk-inducing effect. The magnitude is large, at between 24\% and 30\%. However, the matching plots reveal that there is frequently a relatively significant characteristic difference between the control and treated groups. For example, all of {\it Cash Gen / Cash Reqd}, {\it Interest} and {\it Fin Lev} show a significant lack of overlap. Work is required here to establish why that is. }
\subsection{STOXX Eastern Europe 300}
\subsubsection{Independent Chairperson or Female CEO}
{Mentioned earlier in this chapter was the difficulty in attaining a sufficiently high quality matching rate for the STOXX Eastern Europe 300 index. This is reflected by the presence of a single results set here, which despite these difficulties actually has good matching quality with significant x-axis overlap. This result set uses the presence of an independent chairperson or female CEO as the treatment, and the Altman Z score as the outcome indicator. This is based on statement \ref{eastThree}, where \cite{moldovan2015learning} state that the presence of such a condition is positively related to safe bankruptcy risk levels. The findings here support this correlation, with a strong ATE of between 24\% and 34\% in the positive direction. }
